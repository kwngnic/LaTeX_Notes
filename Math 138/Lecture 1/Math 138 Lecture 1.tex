\documentclass[12pt]{letter}
\usepackage{amsmath, amssymb}
\usepackage{fullpage}
\begin{document}
\begin{center}{{\LARGE \textbf{Math 138 - January $4^{th}$ 2015}} \\  \underline{Review of Integration}}\end{center}

Given a, b {\Large $\epsilon$} $\mathbb{R}$. a $<$ b and n {\Large $\epsilon$} $\mathbb{N}$ we divide the interval [a, b] into n subintervals [x$_{i-1}$, $x_{i}$],
i = 1 ... n of equal width $\Delta x$ = $\frac{b - a}{n}$. amd choose sample points:  \\ 
\begin{center}$x_{i} * \epsilon [x_{i-1}, x_{i}]$, i = 1 ... n\end{center} 
For a function $f$ on the closed interval [a, b]. the definite integral of  f from a to b is: \\
{\large{$\int_{a}^{b} f(x) dx$ = $\lim{n\to\infty} \sum_{i=1}^n f(x_{i} *) \Delta x$}} \\  \\
We say that $f$ is integrable if the above limit exists and is independant of the sample points $x_{i} *$. \\
Recall that continuous functions are integrable. \\
 Furthermore, $\lim{n\to\infty} \sum_{i=1}^n f(x_{i} *) \Delta x$ is the sum of the areas of the rectangles, and $\int_{a}^{b} f(x) dx$ is the area under a curve between two $x$ values. \\ \\
\underline {Fundamental Theorem of Calculus}
Suppose $f$ is continous on [a,b]. \\ \\
I: If $g(x) =  \int_{a}^{x} f(t) dt$ , then g is differentiable and $g'(x) = f(x), x${\Large $\epsilon$} [a, b] \\ \\
II: $\int_{a}^{b} f(x)dx = F(b) - F(a)$ where $F$ is \textbf{any} antiderivative of $f(x)$, i.e, $F'(x) = f(x), x$ {\large {$\epsilon$}} [a, b]

\underline{Examples:} \\ 
1) Find the derivative of g(x), where: \\ a) $g(x) = \int_{1}^{x} ln(1+t^{2})dt$ \\ b)  $g(x) = \int_{1}^{3x+2}\frac{t}{1+t^{3}}dt$ \\ \\
Solution a): $g'(x) = ln(1+x^{2})$ By FTC(I) \\
Solution b): Write $g(x) = f(u(x))$, where $f(u) = \int_{1}^{u}\frac{t}{1+t^{3}}dt$ and $u(x) = 3x + 2$
By the chain rule and FTC(I), \\ \begin{equation} \begin{split} g'(x) &= f'(u(x)) \cdot u'(x) \\ &= \end{split} \end{equation}
\end{document}